\documentclass[12pt, a4paper, oneside]{book}
%\author{aya}
\usepackage[top=3cm,right=2cm,bottom=3cm,left=2cm]{geometry}
\usepackage{graphicx}
\usepackage{indentfirst}
\usepackage{anyfontsize}
\usepackage{wrapfig}
\usepackage{enumerate,letltxmacro}
\usepackage[english]{babel}
\LetLtxMacro\itemold\item
\graphicspath{{images/}}

\setlength{\parindent}{4em}
\renewcommand{\item}{\itemindent4.5em\itemold}
\addto\captionsenglish{\renewcommand{\chaptername}{Bab}}
\addto\captionsenglish{\renewcommand{\contentsname}{Daftar Isi}}
\addto\captionsenglish{\renewcommand{\listfigurename}{Daftar Gambar}}


\begin{document}
	\begin{center}
			\fontsize{16}{24}{
				\textbf{
					Pemrograman Berbasis Kerangka Kerja\\
					Sistem Jaminan Kesehatan\\
		}
	}
	\end{center}
	\begin{center}
		\includegraphics[width=0.5\textwidth]{lambangits}
		\linebreak
			\begin{table}[h]
				\centering
				\begin{tabular}{lr}
					M. Vijay Fathur       & 5112100043 \\
					Nabila Tsurayya Silmi & 5112100055 \\
					Faishal Azka J        & 5112100061 \\
					R. M Iskandar Z       & 5112100101 \\
					Reyhan Arief          & 5112100175 \\
				\end{tabular}
			\end{table}
		\end{center}
		\begin{center}
		{		
			\fontsize{16}{24}{
				\textbf{
					Fakultas Teknologi Informasi\\
					Jurusan Teknik Informatika\\			
					Institut Teknologi Sepuluh Nopember Surabaya\\
				} 
			}
		}
	\end{center}
	
	\thispagestyle{empty}
	\pagebreak
	\tableofcontents
	\pagebreak
	\listoffigures
	\pagebreak
	\chapter{Deskripsi Produk}
	\section{Perspektif Produk}
		Produk yang yang hendak dikembangkan adalah sistem pendukung kesehatan, yaitu Sistem Asuransi Kesehatan.\par
		{Tujuan dibuatnya Sistem Asuransi Kesehatan antara lain, adalah sebagai berikut:
		\begin{enumerate}
			\item Mempermudah proses pendaftaran anggota asuransi kesehatan.
			\item Mempermudah administrasi asuransi kesehatan.
			\item Mempermudah anggota untuk mendapatkan informasi lokas fasilitas kesehatan.
		\end{enumerate}}
		\par
		Diagram Konteks
	\section{Klasifikasi dan Karakteristik Pengguna}
		\begin{table}[h]
		\centering
		\begin{tabular}{|c | c | l |}
			\hline
			No	& Kategori Pengguna	& Hak Akses ke Aplikasi\\ 						
			\hline
			1.	& Calon Anggota		& 1. Mendaftarkan diri sebagai anggota.\\			
			\hline
			2.	& Anggota			& 1. Melihat fasilitas kesehatan terdekat.\\		
				&					& 2. Melihat status iuran.\\						
				&					& 3. Melihat info riwayat kesehatan\\
			\hline
			3. 	& Administrator		& 1. Melihat data anggota\\
				&					& 2. Mengelola data\\
			\hline
		\end{tabular}
	\end{table}
	\section{Lingkungan Pengoperasian}
		Lingkungan operasi untuk menjalankan Sistem Asuransi Kesehatan adalah sebagai berikut:
\begin{enumerate}[1.]
	\item Bahasa Pemrograman yang akan digunakan yaitu Java dengan kakas kerangka kerja sebagai berikut:
	\begin{enumerate}[a.]
		\item Maven
		\item Spring Framework
		\item Hibernate
	\end{enumerate}
	\item Perangkat Lunak yang digunakan untuk membangun Sistem ini adalah \textit{Eclipse IDE for Java EE}
\end{enumerate}
	\section{Batasan dan Desain Implementasi}
		Sistem Jaminan Kesehatan akan dibangun menggunakan beberapa perangkat, seperti:
\begin{enumerate}[1.]
	\item Menggunakan \textit{Eclipse for Java EE}
	\item Menggunakan \textit{framework} yaitu:
	\begin{enumerate}[a.]
		\item Maven. 
		\item Spring \textit{Framework}.
		\item Hibernate.
		\item Axis2.
	\end{enumerate}
\end{enumerate}
\par
\\
Batasan dan Desain Implementasi Sistem Jaminan Kesehatan adalah sebagai berikut:
\begin{enumerate}
	\item Terintegrasi denga sistem pokok kependudukan dan sistem pokok kesehatan.
	\item Sistem berbasis web sehingga membutuhkan jaringan internet untuk mengakses sistem.
\end{enumerate}
	\section{Asumsi dan Ketergantungan}
		Sistem Jaminan Kesehatan ini memerlukan data yang diambil dari sistem lain. Berikut data yang akan di ambil dari sistem lain.
\begin{enumerate}[1.]
	\item Data yang diambil dari Sistem Pokok Kependudukan. 
	\begin{enumerate}[a.]
		\item NIK.
		\item ID KK.
	\end{enumerate}
	\item Data yang diambil dari Sistem Pokok Kesehatan.
	\begin{enumerate}[a.]
		\item Data Rumah Sakit.
		\item Data Tenaga Medis.
	\end{enumerate}
\end{enumerate}
	\pagebreak
	\chapter{Fitur}	
	\section{Keanggotaan Jaminan Kesehatan}
	\subsection{Deskripsi Produk}
		Fitur ini digunakan oleh calon anggota BPJS Kesehatan untuk mendaftar menjadi anggota BPJS Kesehatan. Calon anggota BPJS Kesehatan dapat mendaftarkan diri sendiri dan seluruh anggota keluarga hanya dengan Kartu Keluarga. Jika ada anggota keluarga baru yang belum terdaftar, maka dapat mendaftar dengan memasukkan nomor KTP dari pendaftar dan nomor Kartu Keluarga.
	\subsection{Kebutuhan Fungsional}
	\section{title}
	\subsection{Deskripsi Produk}
	\subsection{Kebutuhan Fungsional}
	\end{document}
\end{document}
