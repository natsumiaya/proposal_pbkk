\documentclass[12pt, a4paper, oneside]{book}
%\author{aya}
\usepackage[top=3cm,right=2cm,bottom=3cm,left=2cm]{geometry}
\usepackage{graphicx}
\usepackage{indentfirst}
\usepackage{anyfontsize}
\usepackage{wrapfig}
\usepackage{enumerate,letltxmacro}
\usepackage[english]{babel}
\LetLtxMacro\itemold\item
\graphicspath{{images/}}

\setlength{\parindent}{4em}
\renewcommand{\item}{\itemindent4.5em\itemold}
\addto\captionsenglish{\renewcommand{\chaptername}{Bab}}
\addto\captionsenglish{\renewcommand{\contentsname}{Daftar Isi}}
\addto\captionsenglish{\renewcommand{\listfigurename}{Daftar Gambar}}


\begin{document}
	\begin{center}
			\fontsize{16}{24}{
				\textbf{
					Pemrograman Berbasis Kerangka Kerja\\
					Sistem Jaminan Kesehatan\\
		}
	}
	\end{center}
	\begin{center}
		\includegraphics[width=0.5\textwidth]{lambangits}
		\linebreak
			\begin{table}[h]
				\centering
				\begin{tabular}{lr}
					M. Vijay Fathur       & 5112100043 \\
					Nabila Tsurayya Silmi & 5112100055 \\
					Faishal Azka J        & 5112100061 \\
					R. M Iskandar Z       & 5112100101 \\
					Reyhan Arief          & 5112100175 \\
				\end{tabular}
			\end{table}
		\end{center}
		\begin{center}
		{		
			\fontsize{16}{24}{
				\textbf{
					Fakultas Teknologi Informasi\\
					Jurusan Teknik Informatika\\			
					Institut Teknologi Sepuluh Nopember Surabaya\\
				} 
			}
		}
	\end{center}
	
	\thispagestyle{empty}
	\pagebreak
	\tableofcontents
	\pagebreak
	\listoffigures
	\pagebreak
	\chapter{Deskripsi Produk}
	\section{Perspektif Produk}
		\input{perspektif_produk}
	\section{Klasifikasi dan Karakteristik Pengguna}
		\begin{table}[h]
		\centering
		\begin{tabular}{|c | c | l |}
			\hline
			No	& Kategori Pengguna	& Hak Akses ke Aplikasi\\ 						
			\hline
			1.	& Calon Anggota		& 1. Mendaftarkan diri sebagai anggota.\\			
			\hline
			2.	& Anggota			& 1. Melihat fasilitas kesehatan terdekat.\\		
				&					& 2. Melihat status iuran.\\						
				&					& 3. Melihat info riwayat kesehatan\\
			\hline
			3. 	& Administrator		& 1. Melihat data anggota\\
				&					& 2. Mengelola data\\
			\hline
		\end{tabular}
	\end{table}
	\section{Lingkungan Pengoperasian}
		\input{lingkungan_pengoperasian}
	\section{Batasan dan Desain Implementasi}
		\input {batasan}
	\section{Asumsi dan Ketergantungan}
		\input{asumsi_ketergantungan}
	\pagebreak
	\chapter{Fitur}	
	\section{Keanggotaan Jaminan Kesehatan}
	\subsection{Deskripsi Fitur}
		Fitur ini digunakan oleh calon anggota BPJS Kesehatan untuk mendaftar menjadi anggota BPJS Kesehatan. Calon anggota BPJS Kesehatan dapat mendaftarkan diri sendiri dan seluruh anggota keluarga hanya dengan Kartu Keluarga. Jika ada anggota keluarga baru yang belum terdaftar, maka dapat mendaftar dengan memasukkan nomor KTP dari pendaftar dan nomor Kartu Keluarga.
	\subsection{Kebutuhan Fungsional}
		Berikut adalah kebutuhan fungsional keanggotaan jaminan kesehatan:
\begin{enumerate}
	\item Mendaftar sebagai Anggota Jaminan Kesehatan menggunakan nomor Kartu Keluarga.
	\item Mendaftar sebagai Anggota Jaminan Kesehatan menggunakan nomor Kartu Keluarga dan nomor KTP (Jika calon anggota merupakan anggota keluarga baru).
\end{enumerate}
	\section{Penggunaan Layanan Kesehatan}
	\subsection{Deskripsi Fitur}
		Fitur ini digunakan untuk mencatat layanan kesehatan apa saja yang telah digunakan oleh anggota BPJS Kesehatan.
	\subsection{Kebutuhan Fungsional}
		Berikut adalah kebutuhan fungsional penggunaan layanan kesehatan:
\begin{enumerate}
	\item Menunjukkan pengeluaran penggunaan layanan kesehatan anggota BPJS Kesehatan dari suatu fasilitas kesehatan.
\end{enumerate}
	\section{Iuran Jaminan Kesehatan}
	\subsection{Deskripsi Fitur}
		Fitur ini digunakan untuk mencatat iuran (premi) jaminan kesehatan.
	\subsection{Kebutuhan Fungsional}
		Berikut adalah kebutuhan fungsional iuran jaminan kesehatan:
\begin{enumerate}
	\item Menunjukkan histori pembayaran iuran jaminan kesehatan kepada anggota BPJS Kesehatan.
	\item Memasukkan data pembayaran iuran jaminan kesehatan.
\end{enumerate}
	\section{Rujukan}
	\subsection{Deskripsi Fitur}
		Fitur ini digunakan oleh Instansi Kesehatan untuk merujuk anggota BPJS Kesehatan yang membutuhkan pengobatan lebih lanjut ke fasilitas kesehatan lainnya.
	\subsection{Kebutuhan Fungsional}
		Berikut adalah kebutuhan fungsional rujukan:
\begin{enumerate}
	\item Menampilkan daftar fasilitas kesehatan yang ada di kota di mana anggota BPJS Kesehatan sedang berobat.
	\item Merujuk anggota BPJS Kesehatan yang membutuhkan pengobatan lebih lanjut ke fasilitas kesehatan lainnya.
\end{enumerate}
\end{document}
